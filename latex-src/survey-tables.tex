%----------------------------------------------------------------
%
%  File    :  survey-recc.tex
%
%  Author  :  Mirza Kabiljagic, Stefan Rajinovic, Aleksandar Stojicic,
%             Inti Gabriel Mendoza Estrada
%
%  Created :  27 May 20xx
%
%  Changed :  4 Dez 2019
%
%----------------------------------------------------------------



\chapter{HTML Tables}

\label{HTML5 Tables}

In case of having too much data to display, or data which is
inherently better in a grid, HTML tables are the best solution
available. Before, tables have been used mostly for the layout of HTML
websites (and is now an example of bad website development pracices)
mainly due to two reasons:
\begin{itemize}
    \item[--] Semantically, it is wrong.
    \item[--] Tables aren't as adaptable and flexible as divs'
\end{itemize}

\section{Structure of Tables}

The basic structure of an HTML table starts with the \elname{<table>}
tag. It is the starting point for constructing a table. Now, HTML
tables consist of columns and rows, like normal tables. For rows, the
tag  \elname{<tr>} is used, whereas for table header \elname{<th>} tag
is used. Normally, table headers are positioned in the center and are
bold. For table cells \elname{<td>} tag is used \parencite{AdamWood}.

\begin{lstlisting}[%
    float = tp,
    language = CSS,
    xleftmargin=0cm,              % no extra margins for floats
    xrightmargin=0cm,             % no extra margins for floats
    language=biblatex,
    basicstyle=\footnotesize\ttfamily,
    frame=shadowbox,
    numbers=left,
    label=list:Structure,
    stringstyle=\color{blue}
    ,
    caption={[Structure ]
    },
]
% An example of an HTML Table which demonstrates information about cars:
<!DOCTYPE html>
<html>
<head>
  <title>Best Cars 2019</title>
</head>
<body>
<table class="table table-bordered table-hover table-condensed">
  <thead>
    <tr>
      <th title="Field #1">Car</th>
      <th title="Field #2">Manufacturer</th>
      <th title="Field #3">Engine Size</th>
      <th title="Field #4">Cylinders</th>
      <th title="Field #5">Horsepower</th>
      <th title="Field #6">Torque</th>
      <th title="Field #7">Compresion Ratio</th>
      <th title="Field #8">Miles per gallon</th>
      <th title="Field #9">Price</th>
    </tr>
  </thead>
  <tbody>
    <tr>
      <td>2019 Acura RDX</td>
      <td>Acura</td>
      <td>2.00L</td>
      <td align="right">4</td>
      <td align="right">272</td>
      <td align="right">280</td>
      <td>9.8:1</td>
      <td align="right">28</td>
      <td>€33,600.00</td>
    </tr>
    <tr>
      <td>2019 Ford Ranger</td>
      <td>Ford</td>
      <td>2.30L</td>
      <td align="right">4</td>
      <td align="right">270</td>
      <td align="right">310</td>
      <td>10.0:1</td>
      <td align="right">21</td>
      <td>€21,800.00</td>
    </tr>
  </tbody>
</table>

</body>
</html>

\end{lstlisting}

Now, there exist more tags which can be used for HTML5 tables:
\begin{itemize}
    \item[--] $<$\elname{thead}$>$ - Table header, it is used to point
out single or multiple rows of a table, which do not contain table
data but column labels \parencite{ChrisCoyier}.

    \item[--] $<$\elname{tbody}$>$ - Table body, it is used to point
out $<$\elname{tr}$>$ elements. Position this tag always after
$<$\elname{thead}$>$, but it can also come after or before
$<$\elname{tfoot}$>$ \parencite{ChrisCoyier}.

    \item[--] $<$\elname{tfoot}$>$ - Table footer, it is used to point
out single or multiple $<$\elname{tr}$>$ elements where those elements
are presenting an overview  of the data in the table
\parencite{ChrisCoyier}.

    \item[--] $<$\elname{caption}$>$ - Table caption, as the name
already says, can be used to specify table caption. Can be put on the
bottom of the CSS document.

    \item[--] $<$\elname{col}$>$ - While using col and some other
keyword, for example, align, it is possible direct the alignment of
text in the table. There are other keywords whom can be used to adjust
colors, width and many other things of table columns.
\end{itemize}

An example HTML table code can be seen in Listing \ref{list:Structure}.
